\chapter{Marco referencial}
\label{sec:introduction}

\section{Introducción}
Esta sección presenta aquellos antecedentes, tanto teóricos como prácticos, que permiten identificar la importancia de su trabajo. La introducción es desarrollada inicialmente durante la presentación de perfil de proyecto, sin embargo, se ve complementada al finalizar el trabajo; cuando se tiene claro el aporte del mismo. Ciertamente, una vez terminado su trabajo, puede incluir en la introducción una breve reseña de los resultados alcanzados u obtenidos; véase la lista de ejemplos. 

La extensión de la introducción no debe sobrepasar las dos páginas y se recomienda un mínimo de una página. Es usual que el contenido emplee muchas referencias, en especial para citar rápidamente anteriores soluciones o las fuentes que dan validez al escenario en el que se encuentra el problema. Siempre debe tomarse en cuenta que el lector espera que la introducción sea atractiva, concisa y no redundante. 

Si bien la introducción puede señalar puntos problematizadores, estos no deben ser desarrollados ampliamente; los puntos problematizadores se desarrollan en pleno durante el planteamiento del problema. Una forma de reconocer que los puntos problematizadores se tratan muy profundamente durante la introducción es por la extensión: si la introducción es muy extensa y existe sólo un tópico, es decir, es repetitiva, probablemente la sección se está confundiendo con el planteamiento del problema. 

\section{Planteamiento del Problema}
Consiste del preámbulo a la definición del problema y puede ser entendido como la descripción del escenario en el cual se encuentra el nodo o punto problematizador, i.e. donde se encuentra la necesidad o dificultad que se desea abordar. En ese sentido, en este punto se proporcionan todos aquellos datos que permiten luego converger a la definición del problema.
Dado que se entiende los problemas no son únicos y han sido abordados por distintas personas con diferentes perspectivas, el planteamiento del problema incluye una reseña de los problemas de anteriores soluciones; reiterando, el énfasis se aboca a los problemas. La introducción, por otra parte, pudo ya haber incluido la mención de dichas soluciones previas, pero esto no significa que el texto de esa sección sea la misma de esta. En efecto, el planteamiento del problema se enfoca en las dificultades, necesidades o puntos problematizadores, y la introducción se orienta hacia la presentación del tema. 

\subsection{Definición del Problema}
Este punto comprende un párrafo que permite la lectura del problema puntual y concreto. Dicho párrafo debe ser el punto final del planteamiento y al mismo tiempo el punto que permite la identificación de requerimientos y objetivos de diseño. 

\section{Objetivos}
\subsection{Objetivo general}
Es entendido como la premisa general que el estudiante realizará para resolver el problema. El objetivo sólo puede ser formulado iniciando con un verbo en infinitivo.

Se entiende que a través del objetivo se contribuye a solucionar el problema, por lo que, existe una conexión inherente y profunda entre ambos. Por otro lado, este objetivo tiene una relación directa con el título siendo que el título sintetiza el trabajo en una denominación. 

Estructuralmente, el objetivo general inicia con un objetivo en infinitivo, a lo cual le sigue un objeto de estudio y finaliza con la mención del objeto de atención del proyecto.

Asumiendo el sistema que será propuesto posee distintas funcionalidades, cada objetivo específico o varios de ellos pueden ser formulados para describir el cómo se llegará a desarrollar cada una de ellas. En ese sentido, es usual encontrar que exista secuencialidad entre los objetivos.

\begin{itemize}
	\item Punto 1
	\item Punto 2
	\item Punto 3
\end{itemize}
