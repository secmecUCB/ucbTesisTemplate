\chapter{Marco teórico}
\label{sec:teoria}
\section{Estado del Arte }
De manera sencilla, el estado del arte puede ser entendido como el conjunto más reciente, o actual, de conocimientos y procedimientos referidos a un tema en específico. Desde otra perspectiva, uno puede indicar que se trata de aquella tecnología de punta que caracteriza al tópico. En ambos casos, la definición acierta en indicar su naturaleza moderna y de innegable trascendencia; el estado del arte marca una tendencia de desarrollo en las comunidades de investigación. 

Un ejemplo, cuando el tópico específico es la “clasificación automática de objetos por medio de computador”, el estado del arte evidente es conformado por los métodos basados en redes neuronales profundas y convolucionales. La razón es clara, estos procedimientos permiten alcanzar los mayores índices de clasificación automática de imágenes. 
Considerando que cada trabajo de grado aborda una temática, es posible determinar que existe un conjunto de conocimientos y procedimientos caracterizados por ser los más actuales, es decir, existe un estado del arte para cada trabajo de grado. En ese sentido, esta sección tiene el propósito de presentar aquellas soluciones previas más recientes que forman parte del estado del arte propio del trabajo. 

La revisión de soluciones previas se distingue por ser un proceso imprescindible dentro la investigación y el diseño ingenieril. Esto se debe a que permite al diseñador, y en este caso estudiante, situarse dentro un espacio de diseño coherente, moderno y significativo. Por tanto, una revisión adecuada debe contemplar múltiples soluciones previas de calidad. 

\subsection{Consideraciones adicionales}
La cantidad de enfoques es variable, pero al menos son necesarios dos. En ese sentido, se espera que la sección sea desarrollada en al menos dos subsecciones. Por otro lado, con relación a la cantidad de referencias requeridas para considerar que la revisión fue exhaustiva, ésta es de al menos treinta. Esta cantidad corresponde tanto a la perspectiva académica como a la del mercado, siendo al menos diez la cantidad de trabajos académicos.

Claro está, aunque no está demás indicar, que los documentos a referenciar deben ser de alta calidad; i.e. haber sido publicados en sitios cuya política de revisión sea la de pares. Asimismo, los productos del mercado citados deben ser estimados como aquellos que caracterizan la industria o la oferta de máquinas; de distintos proveedores. 
 
Finalmente, antes de pasar a los enfoques, es posible incluir texto de introducción al estado del arte. Esto supone un párrafo que explica la forma en la cual se desarrollará el estado del arte.
